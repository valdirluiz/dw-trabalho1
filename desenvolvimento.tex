% Desenvolvimento

\chapter{Planejamento}

\section{Definição da equipe}

Os recursos humanos e seus respectivos papéis 
envolvidos no projeto são definidos na Tabela \ref{tab:recursos} . 

\begin{table}[!htb]
    \begin{center}
        \caption{Recursos Humanos.} \label{tab:recursos}
        \begin{tabular}{ p{4cm} | p{4cm} }
            \hline
            \textbf{Papel} & \textbf{Recurso Humano} \\
            \hline
            Diretor de DW & Ronan Romeu Knob \\
            \hline
            Gerente de projetos de DW & Sabrina Schütz de Oliveira
             \\
            \hline
            Analista de negócios & Giancarlo Souza de Freitas \\
            \hline
            Arquiteto de DW & Giancarlo Souza de Freitas \\
            \hline
            Equipe técnica & Valdir Luiz Hofer Arnhold \\
            \hline
        \end{tabular}
    \end{center}
    Fonte: Os autores (2017).
\end{table}
 
\section{Cronograma}

O projeto de \textit{Data Mart} Ronbuster 
foi pensado para ter início no mês 
de março. Sua duração estimada foi 
estipulada em 6 meses, contemplando as 
fases apresentadas no tabela x. 
A escolha deste período de início foi 
devido a que é o período de término das 
férias escolares, o que deve baixar o movimento 
das filiais, estando pronto também antes do final 
do ano, novo grande ciclo de férias.

Sendo assim, as datas iniciais de início e término do projeto são:

\begin{itemize}
    \item Início do projeto - 01/03/2017;    
    \item Término do projeto - 31/08/2017;    
    \item Dias úteis neste período: 129; e
    \item Horas do projeto: 1032h
\end{itemize}

Abaixo temos uma estimativa das fases do projeto, e seu tempo esperado.
Os valores apresentados não consideram feriados e sábados e domingos. Cálculo da hora é feito por dias úteis * 8h.

\begin{table}[!htb]
    \begin{center}
        \caption{Estimativa de tempo.} \label{tab:tempo}
        \begin{tabular}{ p{3cm} |  p{2.2cm} | p{2.3cm} | p{1.2cm} }
            \hline
            \textbf{Fase} & \textbf{Data de início}  & \textbf{Data de término} & \textbf{Duração} \\
            \hline
            Planejamento do projeto & 01/03/20177 & 24/03/2017 & 144 h \\
            \hline
            Entrevistas &
            27/03/2017 &
            07/04/2017 &
            80 h \\
            \hline
            Definição do esquema estrela e plano de ação &
            10/04/2017 &
            20/04/2017 &
            64 h \\
            \hline        
            Definição da equipe do DW &
            24/04/2017 &
            26/04/2017 &
            24 h \\
            \hline 
            Implementação do DW &
            27/04/2017 &
            09/06/2017 &
            248 h \\
            \hline            
            Criação de portal de dashboards e relatórios &
            12/06/2017 &
            04/08/2017 &
            320 h \\
            \hline 
            Treinamento &
            07/08/2017 &
            11/08/2017 &
            40 h \\
            \hline
            Homologação e ajustes pontuais &
            14/08/2017 &
            31/08/2017 &
            112 h \\
            \hline
            \multicolumn{3}{r|}{Total} & 1032 h \\
            \hline

        \end{tabular}
    \end{center}
    Fonte: Os autores (2017).
\end{table}


\section{Custos}

Com base no Cronograma mostrado e nas horas estimadas o custo projeto é de R\$ 52.000,00.

\chapter{MODELAGEM DIMENSIONAL}

Para a construção do modelo dimensional, 4 passos são necessários, sendo eles:

\begin{enumerate}
\item Decisão de qual processo de negócio deve-se modelar, com base na combinação do conhecimento do negócio com o conhecimento dos dados disponíveis.
\item Definição do grão do processo do negócio, o qual é o nível fundamental atômico de dados que representará o processo na tabela de fatos.
\item Escolha das dimensões que serão aplicadas a cada registro da tabela de fatos.
\item Escolha dos fatos mensuráveis que irão popular cada registro da tabela de fatos.
\end{enumerate}

A seguir, temos a utilização desses passos para a criação da modelagem dimensional em questão, para auxiliar a resolução das perguntas estratégicas levantadas.
 
\section{processo do negócio}

O processo do negócio que será modelado é o movimento diário de item, nos permitindo acompanhar quais filmes estão sendo emprestados, em que lojas, a que preço e em que dias.


\section{granularidade}

A granularidade, representa o nível de
detalhamento da modelagem.
Para este projeto, se buscou alcançar o maior nível de granularidade possível. Por exemplo, para o 
mapeamento da dimensão do tempo é utilizado a hora, 
dessa forma se torna possível fazer 
consultas por dia ou semanas. 

\section{dimensões}

As dimensões escolhidas são Tempo, Midia, Loja e Cliente.  

\section{fatos}