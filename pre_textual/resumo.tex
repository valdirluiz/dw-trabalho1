% Resumo

\textoResumo{
    Um agente cognitivo artificial utiliza bases de conhecimentos sobre as quais faz inferências buscando tomar decisões. 
    Com o crescimento da produção de informação, gerando consequentemente conhecimento, é preciso que agentes inteligentes sejam capazes de armazenar e raciocinar sobre diferentes formas de conhecimento. 
    Além disso, a informação e o conhecimento são dinâmicos, dessa forma o agente também deve permitir que novas formas de representação de conhecimento sejam adicionadas ao seu ciclo de raciocínio. 
    Com essas premissas, o presente trabalho propõe a implementação de um interpretador para o desenvolvimento de agentes com base em sistemas multi-contexto para mapear o conhecimento. 
    Buscando possibilitar a integração de diferentes formas de conhecimento no ciclo de raciocínio. 
}
\palavrasChave{
    Agentes.
    Sistemas multi-contexto.
    Interpretador.
}

\paginaresumo
